\begin{question}

    \begin{enumerate}[label=\textbf{\alph*})]
		\item Os elementos de baixa frequência são aqueles que possuem 
		baixa variabilidade na mudança do tons de cinza na imagem(homogênea), então isso seria 
		a região onde tem o fundo preto.
  		\item Os elementos de alta frequência seriam aqueles onde há uma brusca modificação 
        do nível do tom de cinza na imagem. Isto seria as bordas e o ruído.
    	\item Para obter a imagem B a partir de A, foi aplicado um filtro da média. Pois a imagem 
		teve seus elementos de alta frequência suavizados
        \item O filtro de mediana foi passado para obter a imagem C, o ruído de alta frequência foi 
        suavizado com as bordas preservadas
        \item Por fim, a imagem D é o resultado do filtro de Sobel para detecção de bordas, o que 
        pode explicar os elementos de alta frequência realçados e os de baixa frequência suavizados.
	\end{enumerate}
    
\end{question}