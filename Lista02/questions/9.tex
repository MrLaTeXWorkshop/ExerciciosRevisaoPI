\begin{question}

    \begin{enumerate}[label=\textbf{\alph*})]
		\item Os elementos de baixa frequência são aqueles que possuem 
		baixa variabilidade na mudança do tons de cinza na imagem, então isso seria 
		a região onde tem o fundo preto e as regiões do cérebro com cor homogênea por exemplo.
  		\item Os elementos de alta frequência seriam aqueles onde há uma brusca modificação 
        do nível do tom de cinza na imagem. Isto seria por exemplo a transição do fundo preto 
		para o cérebro e da região central do cérebro onde vai do branco para o cinza. 
    	\item Para obter a imagem B a partir de A, foi aplicado uma convulação de uma máscara 
        de blur gaussiano, isso é perceptível pela perca de detalhes, além do fato que 
		as bordas estão consideravelmente mais borradas.
        \item Para obter a Imagem C, foi aplicado o desfoque que gerou a imagem B, e então 
        foi aplicado um operador de Laplace ( também conhecido como Sharpen ). Isto pode ser notado 
		pelo fato que a imagem C é notavelmente mais nítida que a imagem B, porém não possui 
		o nível de detalhes que o crânio tem na Imagem A. Em compensação, o ruído foi removido
        \item Por fim, a imagem D é o resultado do filtro de Sobel para detecção de bordas, o que 
        pode explicar porque a imagem tem apenas dois tons, preto para indicar o fundo, e branco 
		para detectar as bordas, incluindo as bordas dos ruídos da imagem original. 
	\end{enumerate}
    
\end{question}