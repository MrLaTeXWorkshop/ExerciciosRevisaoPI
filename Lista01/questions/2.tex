\begin{question}

  A amostragem é o processo de converter a imagem analógica em 
  uma matriz MxN pontos, onde cada ponto é um pixel. 
  Já a quantitização é o processo onde cada um dos pixels da imagem, assumam 
  um valor inteiro entre 0 a $2^{n}-1$. O valor \textbf{n} representa o 
  número de níveis de cinza presentes na imagem digitalizada.

\end{question}
